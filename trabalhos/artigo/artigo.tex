\documentclass[12pt]{article}

\author{João Vitor Maia Neves Cordeiro}
\title{Aplicações de machine learning em IoT: uma visão sobre o estado da arte}
\date{\today}

\begin{document}

\maketitle

\begin{abstract}
O presente trabalho visa fomentar um debate sobre o estado da arte das aplicações de machine learning (ML) em tecnologias Internet of Things (IoT), a fim de dar ao leitor o material necessário para compreender a importância da multidisciplinaridade entre as mais diversas áreas da computação;

O artigo está dividido da seguinte forma: uma breve introdução sobre os temas ML e IoT; uma série de exemplos práticos da interligação entre as duas áreas estudadas; e por fim uma discussão sobre o que há de mais avançado sendo desenvolvido em universidade e indústrias pelo mundo;

Como metodologia para a coleta de dados foi realizada uma pesquisa bibliográfica buscando materiais publicados nos últimos 3 anos em periódicos e congressos nacionais e internacionais.
\end{abstract}

\section{Introdução}

\subsection{Motivação}

O crescente interesse nos estudos da área de machine learning vem acompanhado de uma certa confusão quanto as reais aplicação dessa tecnologia, sendo disseminada principalmente por publicações equivocadas da grande mídia. Desta forma, é de extrema importância que a comunidade científica formule publicações que tenham como objetivo colaborar com a formação de uma melhor fonte de conhecimento para o público geral.

Da mesma forma, é notável que a expansão do mercado de IoT seja suportada pelo avanço das tecnologias em redes de computadores e mais recentemente pelo aprendizado de máquina.

\subsection{Justificativa}

\subsection{Objetivos Gerais}

Esse artigo se propõe a introduzir alguns conceitos básicos para o melhor entendimento de seu conteúdo, discutir sobre as principais aplicações de técnicas de aprendizado de máquina em IoT bem como apontar avanços notáveis nessa área.

\subsection{Objetivos Específicos}

\begin{itemize}
    \item Introduzir ao leitor os assuntos machine learning e IoT.
    \item Listar de forma não extensiva as aplicações possíveis de técnicas de machine learning em dispositivos IoT
    \item Comentar sobre o estado da arte das já citadas aplicações, com situações reais. 
\end{itemize}

\section{Conceitos básicos}

\subsection{Machine Learning}

O termo machine learning foi cunhado por Samuel Lee Arthur, ainda na década de 50, quando Arthur trabalhava como pesquisador para a IBM desenvolvendo um programa capaz de aprender a jogar xadrez. O algoritmo em questão utilizava dados de outras partidas para prever a probabilidade de vitória a cada ação, tomando o caminho com a maior probabilidade de vitória. O termo se refere a capacidade dos algoritmos de ML de evoluirem conforme a execução, baseando-se em dados para aprender a melhor forma de realizar uma função. 

De forma mais prática, machine learning é área da ciência da computação que estuda algoritmos que podem aprimorar seu funcionamento por base de experiências prévias, sejam essas fornecidas por um banco de dados ou pelos dados que o algoritmo gera durante sua execução. É extensivamente utilizada em situações onde algoritmos convencionais não podem resolver o problema de forma satisfatória ou em tempo hábil.

\subsection{Internet of Things}

A primeira definição do termo \emph{internet of things (IoT)} veio do trabalho de Ashton \emph{et al.} e pode ser traduzido para "uma rede aberta e abrangente de objetos inteligentes que têm a capacidade de se autoorganizar, compartilhar informações, dados e recursos, reagir e agir diante de situações e mudanças no ambiente". Desde então, os "objetos inteligentes" foram renomeados para "coisas" (\emph{things}) e são descritos como dispositivos de hardware e software capaz de se conectar a uma rede com capacidade para performar uma funcionalidade específica. 

Hoje, mais de 20 anos após a primeira definição, os dispositivos IoT invadiram as casas e indústrias do mundo, adquirindo um papel essencial na sociedade moderna e que segundo recente análise [iot] está em crescente tanto em lucro quanto em produtividade. Essas tecnologias podem se espalhar por diversas áreas, sendo as mais notáveis: automação residencial, agricultura inteligente e automação industrial.

\section{Trabalhos correlatos}

\subsection{A Survey of Machine and Deep Learning Methods for Internet of Things
(IoT) Security [1]}

O artigo de Mohammed Ali Al-Garadi \emph{et al}. tem como premissa a
quantidade crescente de dispositivos conectados que operam com
pouca intervenção humana para questionar a importância de se preocupar em dobro com segurança
em um contexto de IoT.

Os dispositivos IoT contam com necessidades específicas da área e muitas
vezes únicas quanto ao escopo de desenvolvimento e aplicação, por consequência disso podem acabar
deixando a desejar quando o assunto é segurança, sendo protegidas por meios comuns e menos modernos de criptografia e controle de acesso. 

Considerando isto, as técnicas de machine learning que obtiveram grandes avanços no passado recente, se
mostram ferramentas capazes de suprir as necessidades de segurança da IoT. O autor mostra, de forma direta e acessível, informações sobre as
tecnologias em machine learning aplicadas a segurança de dispositivos de IoT, criando uma base sólida de conhecimento para
auxiliar no direcionamento de futuras pesquisas do assunto.

\subsection{Crop Management with the IoT: An Interdisciplinary Survey [2]}

O trabalho publicado por Vitali \emph{et al.} começa com uma conceituação histórica sobre a lenta evolução tecnológica da agricultura até a revolução verde, 
que acelerou vertiginosamente a inovação na área. Após isso, os autores citam a introdução dos dispositivos IoT na agricultura como uma segunda revolução verde,
junto com outras novas tecnologias que aos poucos são incorporadas nas lavouras, como \emph{machine learning} e \emph{edge computing}.

São citadas como principais causas dessa proliferação de tecnologias no campo o barateamento de constante de \emph{chips} capazes de executar as aplicações de ML,
o desenvolvimento de novos sensores capazes de desempenhar suas funcionalidades com menos gasto energético e o avanço de técnicas como o \emph{deep learning}.

Os pesquisadores da universidade bolognesa ainda concluem apontando que a ampliação do uso dessas tecnologias em fazendas inteligentes é capaz de otimizar os lucros
de proprietários de terras, assim como de diminuir impactos ambientais causados pelos plantios e aumentar a produção global de alimentos e insumos. Por fim, também fazem
uma projeção de que em poucos anos essas tecnologias estarão nas mãos até de pequenos fazendeiros individuais continuando a melhorar a vida no campo.

\subsection{Realizing an Effective COVID-19 Diagnosis System Based on Machine Learning and IOT in Smart Hospital Environment [3]}

Uma das áreas ainda não citada aqui mas que possui grandes incentivos para pesquisas em IoT são os Smart Hospitals, ambientes de amparo a saúde controlados por dispositivos
interconectados que podem performar diversas funções, desde a captação de sintomas até a administração de medicações. Durante o ano de 2020, essas pesquisas tenderam-se a focar na pandemia
do vírus SARS-COV-2, como é o caso do artigo citado.

No trabalho publicado por Abdulkareem \emph{et al.} nós temos uma pesquisa de cunho mais experimental, lidando diretamente com amostras do vírus. Segundo o texto, foram coletados dados de pacientes
ao redor do mundo com a ajuda de sensores em dispositivos inteligentes.

\subsection{An architecture for adaptive task planning in support of IoT-based machine learning applications for disaster scenarios [4]}

Todos os anos, milhares de desastres naturais e acidentes causados por humanos acontecem pelo mundo, pondo em risco a vida não só das pessoas afetadas mas também dos
profissionais responsáveis por socorrer e resgatar as vítimas dessas situações. O uso de drones para cenários como esses se tornou popular recentemente, devido ao barateamento da tecnologia
utilizada para o funcionamento básico dessas máquinas. O artigo publicado por Sacco \emph{et al.} descreve as a arquitetura de uma possível aplicação de machine learning no aprimoramento desses dispositivos.

Ao alimentar algoritmos de aprendizado de máquina com dados em vídeo captados por drones em desastres é possível treinar um modelo capaz de interpretar em tempo real a situação
de um novo desastre e escolher a melhor ação a ser tomada nesse caso. Em específico, a arquitetura proposta pelo trabalho lida com um cenário de múltiplos agentes semi autônomos
que são programados para realizar tarefas gerais como apagar chamas, remover escombros e carregar vítimas. Após treinado, o modelo é conectado ao drone e passa a monitorar as atividades
que ocorrem no ambiente ao redor dele, pronto para identificar e responder a situações de risco.

A conclusão dos pesquisadores é de que apesar de promissor, o modelo ainda conta com algumas barreiras, como a baixa precisão de sensores em situações tão extremas. É possível que com o avanço
na tecnologia dos sensores, aplicações como essa possam se tornar 100\% autônomas, salvando a vida de milhares de pessoas em desastres, sem arriscar a vida de socorristas.


\end{document}
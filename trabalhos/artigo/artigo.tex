\documentclass{article}

\author{João Vitor Maia Neves Cordeiro}
\title{Aplicações de machine learning em IoT: uma visão sobre o estado da arte}
\date{\today}

\begin{document}

\maketitle

\begin{abstract}
O presente trabalho visa fomentar um debate sobre o estado da arte das aplicações de machine learning (ML) em tecnologias Internet of Things (IoT), a fim de dar ao leitor o material necessário para compreender a importância da multidisciplinaridade entre as mais diversas áreas da computação;

O artigo está dividido da seguinte forma: uma breve introdução sobre os temas ML e IoT; uma série de exemplos práticos da interligação entre as duas áreas estudadas; e por fim uma discussão sobre o que há de mais avançado sendo desenvolvido em universidade e indústrias pelo mundo;

Como metodologia para a coleta de dados foi realizada uma pesquisa bibliográfica buscando materiais publicados nos últimos 3 anos em periódicos e congressos nacionais e internacionais.
\end{abstract}

\section{Introdução}

\subsection{Motivação}

O crescente interesse nos estudos da área de machine learning vem acompanhado de uma certa confusão quanto as reais aplicação dessa tecnologia, sendo disseminada principalmente por publicações equivocadas da grande mídia. Desta forma, é de extrema importância que a comunidade científica formule publicações que tenham como objetivo colaborar com a formação de uma melhor fonte de conhecimento para o público geral.

Da mesma forma, é notável que a expansão do mercado de IoT seja suportada pelo avanço das tecnologias em redes de computadores e mais recentemente pelo aprendizado de máquina.

\subsection{Justificativa}
\subsection{Objetivos Gerais}
\subsection{Objetivos Específicos}

\section{Conceitos básicos}

\section{Trabalhos correlatos}

\end{document}